\documentclass[twocolumn]{aastex631}
\bibliographystyle{aasjournal}

\usepackage{amsmath}
\usepackage{blindtext}
\usepackage{bm}
\usepackage{enumitem}
\usepackage{graphicx}
% Temporary packages
\usepackage{mdframed}
\usepackage{verbatim}

\def\Teff{T_{\rm eff}}
\def\vsini{v\sin{i}}
\def\kmps{\mathrm{km}\;\mathrm{s}^{-1}}

\begin{document}
\title{Blas\'e3D}
\shorttitle{\emph{blas\'e3D}}

\author[0000-0002-2290-6810]{Sujay Shankar}
\author[0000-0002-4020-3457]{Michael Gully-Santiago}
\author[0000-0002-4404-0456]{Caroline V. Morley}
\affil{Department of Astronomy, The University of Texas at Austin, 2515 Speedway, Austin, TX 78712, USA}
\shortauthors{Shankar \& Gully-Santiago \& Morley}

\begin{abstract}
    \blindtext
\end{abstract}

\keywords{}


\section{Introduction}



\subsection{General spectroscopy ideas}

\textbf{what do we want}\\
Key goal: semi-empirical models-- a model that can continuously learn from and adapt to data.  Each new observed data spectrum informs changes to our preconceived theories of how stellar atmospheres produce spectra.  We seek model flexibility where we expect there to be deficiencies in these theories, and rigidity where we expect our theories to be most robust.

\textbf{Why is that hard}\\



Comparison to \citet{czekala15}.  Describe limitation of pre-designating 1 grid point.




blasePaper1 describes a way to overcome this limitation by building up a so-called ``heatmap'' of line properties.  The realization of such a function enables a key capability: semi-empirical models.
This paper implements ths



\begin{mdframed}
    \textbf{Figure: Flowchart-with-words}
\end{mdframed}

\section{Emulating the PHOENIX grid}

\begin{mdframed}
    \textbf{The PHOENIX grid subset}

    \textcolor{lightgray}{\blindtext}
\end{mdframed}

\begin{mdframed}
    \textbf{Cloning with blas\'e}

    - Per-grid point optimization procedure
    - Pretrained model caching
    - Output storage
    \\---------------\\
    Blas\'e's algorithm is designed to take a spectrum as input, and outputs two products:
    1. A list of detected spectral line properties, all assumed to be Voigt profiles, and 2.
    A cloned spectrum, which is the aggregation of all of those detected lines into a spectrum. We
    don't use the cloned spectrum from this process, and only focus on the line properties that are output.
    These two products are stored as a .pt file, which can then be loaded in in order to 
    analyze all the lines in bulk. Each grid point is processed in this way, and the resulting .pt files'
    line properties are aggregated into a single DataFrame, which can then be queried by $T_eff$, $\log(g)$, $Z$,
    in addition to the actual line in question, identified by its center $\mu_j$.
\end{mdframed}

\begin{mdframed}
    \textbf{Open Source Availability}
    \\--------------\\
    The PHOENIX model grid is available open-source at $xx$, and can be interfaced with gollum,
    also available open-source at $xx$. Blas\'e is also available open-source at $xx$, with this work
    existing as a fork named blas\'e3D, available at $xx$.
\end{mdframed}

\begin{mdframed}
    \textbf{Figure: Line density across the grid heatmap}
\end{mdframed}


\section{Line-by-Line Fundamental Stellar Properties}
\begin{mdframed}
    \textbf{Conceptual Illustration: Faceted Plot (Teff, Logg, Z) -> Line Profiles}
\end{mdframed}

\begin{mdframed}
    \textbf{Line Recognition}
    - Need: Identifying unique lines
    - Anticipated friction points: line centroids drift
    - Strategy: (Pre-shift centers)
    \textcolor{lightgray}{\blindtext}
\end{mdframed}

\begin{mdframed}
    \textbf{Line property bulk trends across the grid}
    \textcolor{lightgray}{\blindtext}
\end{mdframed}

\begin{mdframed}
    \textbf{Figure: Heatmap $T_\mathrm{eff}$ vs $\log{g}$ for a single line $2\times2$ panels for $\sigma$, $\gamma$, $A$, $\lambda$ }
\end{mdframed}

\begin{mdframed}
    \textbf{Anomalies in Heatmaps}
    \textcolor{lightgray}{\blindtext}
\end{mdframed}

\section{Mapping Line Parameters to Fundamental Properties}
\begin{mdframed}
    \textbf{Statement: A Bidirectonal Relation exists}
    - Goal: identify a functional form
    \textcolor{lightgray}{\blindtext}
\end{mdframed}

\begin{mdframed}
    \textbf{Figure: Flowchart-with-equations}
\end{mdframed}

\begin{mdframed}
    \textbf{Functional Form}
    - Describe functional form options and refinement procedures
    - List possibilities, Linreg, GPs \citep{2023ARA&A..61..329A}, NNs, etc.
    - We choose LSTSQ
    - Enumerate functional form
    - Adapting model complexity-- AIC
    \textcolor{lightgray}{\blindtext}
\end{mdframed}

\begin{mdframed}
    \textbf{Problem: missing lines}
    - Conceivable Solution 1: treat as NaNs and deal with sparsity
    - Conceivable Solution 2: increase model complexity
    \\--------\\
    When using blas\'e, we aggregate the detected lines from all of the grid points
    into one superset of lines. However, not all of the lines in this superset are
    detected at every grid point. In order to deal with this sparsity, we ignore lines
    have $xx$ detections or less. This removes the most sparse lines, but some sparsity still
    remains. The remaining 'missing lines' are dealt with by treating them as Voigt profiles
    with amplitudes of 0, and Gaussian and Lorentzian widths equal to $xx$ and $xx$ respectively.
\end{mdframed}

\section{Performance evaluation}

Here is a table of the performance of different manifold choices:

\begin{deluxetable*}{cccc}[h]
    \tabletypesize{\small}
    \tablecaption{Notation used in this paper\label{table2}}
    \tablehead{
        \colhead{Model Architecture} & \colhead{Reconstruction RMS} & \colhead{Model Size} & \colhead{Training Time}
        \\\colhead{} & \colhead{RMS \%} & \colhead{MB} & \colhead{x} 
        }
    \startdata
    Linear Polynomial & a & b & c\\
    FFT & a & b & c\\
    Random Forest & a & b & c\\
    Gaussian Process & a & b & c\\
    Interpolation & a & b & c\\
    Interpolation & a & b & c\\ 
    \enddata
\end{deluxetable*}



\begin{mdframed}
    \textbf{Typical Line reconstruction performance}
    \textcolor{lightgray}{\blindtext}
\end{mdframed}

\begin{mdframed}
    \emph{stretch goal}\par
    \textbf{End-to-end PHOENIX grid replication and residual}
    - State the per-pixel residual
    \textcolor{lightgray}{\blindtext}
\end{mdframed}


\section{Discussion}
\begin{mdframed}
    \textbf{Revisiting Model Assumptions}

    \textcolor{lightgray}{\blindtext}
\end{mdframed}

\begin{mdframed}
    \textbf{Limitations}

    - Computational resources
    - Line profile inaccuracy
    - Surface functional form

    \textcolor{lightgray}{\blindtext}
\end{mdframed}


\begin{mdframed}
    \textbf{Conceivable Extensions}
    \\-------------\\
    Future studies can extend this work in several ways:
    \begin{enumerate}[label=-]
        \item Extending the data from just the PHOENIX Grid to additional synthethic spectral
              model grids, such as the Sonora substellar model grids or the COOLTLUSTY substellar model grid.
    \end{enumerate}
\end{mdframed}



\pagebreak
\newpage

\begin{acknowledgments}
    \blindtext
\end{acknowledgments}


\software{}

\bibliography{ms}
\clearpage

\appendix
\section{Notation}
We adopt similar notation to the original blas\'e paper, with some small modifications. We make indexes and
their limits consistent, and remove the use of subscript labeling. We also add new symbols for
wing cut pixels, and radial velocity, as well as introducing three new metrics for reconstruction quality,
storage impact, and computational impact.

\begin{deluxetable}{cp{10cm}}[h]
    \tabletypesize{\small}
    \tablecaption{Notation used in this paper\label{table2}}
    \tablehead{
        \colhead{Symbol} & \colhead{Meaning [former symbol, if applicable]}
    }
    \startdata
    \hline
    \multicolumn{2}{c}{Indexes}\\
    \hline
    $i$ & Pixel index of the precomputed spectrum\\
    $j$ & Spectral line index\\
    $k$ & Grid point index\\
    \hline
    \multicolumn{2}{c}{Spectra}\\
    \hline
    $\bm{\lambda}$ & Spectral axis of the model grid [$\bm{\lambda}_S$]\\
    $\bm{G}$ & The grid of synthetic spectral models\\
    $\mathsf{B}_k$ & Blackbody spectrum of the $k^{th}$ model\\
    $\mathsf{P}_k$ & Continuum polynomial of the $k^{th}$ model\\
    $\mathsf{S}_k$ & Preprocessed flux vector of the $k^{th}$ model\\
    $\mathsf{\hat{S}}$ & Flux vector of the continuous model reconstruction\\
    $\mathsf{R}_k$ & Residual between a $\mathsf{S}_k$ and its reconstruction\\
    \hline
    \multicolumn{2}{c}{Spectral Line Properties}\\
    \hline
    $\mu_{jk}$ & Center position of the $j^{th}$ line in the $k^{th}$ model [$\lambda_c$]\\
    $A_{jk}$ & Amplitude of the $j^{th}$ line in the $k^{th}$ model [$a$]\\
    $\sigma_{jk}$ & Gaussian shape parameter of the $j^{th}$ line in the $k^{th}$ model\\
    $\gamma_{jk}$ & Lorentzian shape parameter of the $j^{th}$ line in the $k^{th}$ model\\
    \hline
    \multicolumn{2}{c}{Scalars}\\
    \hline
    $I$ & Number of pixels in the model grid's spectral axis [$N_x$]\\
    $J$ & Number of spectral lines detected across the grid [$N_{lines}$]\\
    $K$ & Number of grid points in the model grid\\
    $w$ & Wing cut pixel threshold for spectral line identification\\
    $v_r$ & Radial velocity of the star [$RV$]\\
    $R$ & Spectrograph resolving power $\lambda/\delta\lambda$\\
    $Q$ & Quality of a spectral reconstruction\\
    $H$ & The average storage impact of a spectral reconstruction\\
    $Y$ & The average computational impact of a spectral reconstruction\\
    \enddata
\end{deluxetable}

\end{document}

